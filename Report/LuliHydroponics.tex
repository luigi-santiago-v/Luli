\documentclass[12pt]{article}
\usepackage{geometry}
\geometry{a4paper, margin=1in}
\usepackage{graphicx}
\usepackage{float}
\usepackage{titlesec}
\usepackage{lipsum} % For generating dummy text
\usepackage{tocbibind}

\newcommand{\comment}[1]{} % Create a custom comment command
\begin{document}

% Cover Page
\begin{titlepage}
    \centering
    \vspace*{\stretch{1}}
    {\Large\bfseries SENIOR DESIGN PROJECT\par}
    \vspace{1.5cm}
    {\Large ECE 186B\par}
    \vspace{1.5cm}
    {\Large\bfseries Design of Automated Hydroponics with IoT Capabilities\par}
    \vspace{3cm}
    {\large California State University, Fresno\par}
    {\large Lyles College of Engineering\par}
    {\large Electrical and Computer Engineering Department\par}
    \vspace{2cm}
    {\large Dr. Kriehn\par}
    {\large Professor Moore\par}
    \vspace{2cm}
    {\large BY\par}
    \vspace{1cm}
    {\large Liam Goss\par}
    {\large Luigi Santiago-Villa\par}
    \vspace{\stretch{2}}
\end{titlepage}
% Overview
\section*{Overview}
\addcontentsline{toc}{section}{Overview}
The hydroponics system has been successfully developed with a focus on user-friendliness, comprehensive data analytics, and internet connectivity. Functioning seamlessly as a household appliance, the initiative effectively addresses the crucial need for access to nutritious food in areas identified as food deserts. The automated system comprises multiple sensors for monitoring temperature, humidity, light intensity, and water pH, alongside a UV lighting system, water reservoir, water pump, and a microcontroller. These integrated components provide end-users with real-time insights into plant health and nutrient metrics.

Additionally, a web application has been developed to facilitate user interaction, enabling the sharing of insights, posing queries, and exchanging settings presets. Serving as the central nexus for accessing sensor data and engaging with community-driven inputs, this platform significantly enhances user experience. Notably, the solution stands out from existing market offerings by its meticulous design tailored to address the unique challenges of food deserts and foster reliance on local communities through social networking.

The project, spanning two college semesters and totaling nine months, adhered to a budget allocation of approximately \$500. Major milestones were achieved through meticulous planning and execution, encompassing design and implementation phases, further subdivided into research, component sourcing, component testing, and final integration. The resulting fully functional prototype, constructed from Oriented Strand Board (OSB), incorporates the aforementioned sensors, a microcontroller, lighting apparatus, and water management controls. The completion of a fully operational prototype was accomplished early in the project timeline, allowing for the cultivation of basic crops to vividly demonstrate the efficacy and functionality of the design.

\comment{


% Dedication (Optional)
\section*{Dedication}
\addcontentsline{toc}{section}{Dedication}
\lipsum[1] % Replace with your dedication text


}

\comment {


% Acknowledgment (Optional)
\section*{Acknowledgment}
\addcontentsline{toc}{section}{Acknowledgment}
\lipsum[1] % Replace with your acknowledgment text

}

% Table of Contents
\tableofcontents
\pagebreak

% List of Tables
\listoftables
\pagebreak

% List of Figures
\listoffigures
\pagebreak

% Abstract
\section*{Abstract}
\addcontentsline{toc}{section}{Abstract}
This project addresses food deserts by developing an accessible, internet-connected hydroponics system, enabling urban households to grow nutritious food. Leveraging temperature, humidity, light intensity, and water pH sensors, alongside UV lighting and a microcontroller, the system facilitates real-time plant health monitoring. A unique feature is its community-driven web application, allowing users to exchange growing tips and presets, significantly lowering the entry barrier for novices. Designed with a focus on food desert challenges and community engagement, the project aims to make home-grown crops viable for everyone. Over nine months and with a \$500 budget, this initiative will produce a functional prototype that demonstrates efficient crop cultivation, highlighting a technology-driven, socially inclusive approach to mitigate food insecurity in urban areas.


% Narrative Sections (Example structure)
\section{Introduction}
\lipsum[2-3] % Replace with actual content

\subsection{Subsection Example}
\lipsum[4]

\subsubsection{Sub-Subsection Example}
\lipsum[5]

% Figures and Tables example
\begin{figure}[H]
    \centering
    \includegraphics[width=0.8\textwidth]{example-image}
    \caption{Example Figure}
    \label{fig:example}
\end{figure}

\begin{table}[H]
    \centering
    \begin{tabular}{|c|c|c|}
        \hline
        Column 1 & Column 2 & Column 3 \\
        \hline
        Item 1 & Item 2 & Item 3 \\
        \hline
    \end{tabular}
    \caption{Example Table}
    \label{tab:example}
\end{table}

% Conclusion
\section{Conclusion}
\lipsum[6] % Replace with your conclusion

% Appendices
\appendix
\section{Appendix Title}
\lipsum[7] % Replace with your appendices content

% References
\begin{thebibliography}{99}
\bibitem{ref1} Author, A. (Year). Title of the work. Publisher.
% Repeat for each reference
\end{thebibliography}

\end{document}
