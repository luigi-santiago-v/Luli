\documentclass[12pt]{article}
\usepackage{geometry}
\geometry{a4paper, margin=1in}
\usepackage{graphicx}
\usepackage{float}
\usepackage{titlesec}
\usepackage{lipsum} % For generating dummy text
\usepackage{tocbibind}
\usepackage{url}


\newcommand{\comment}[1]{} % Create a custom comment command
\begin{document}

% Cover Page
\begin{titlepage}
    \centering
    \vspace*{\stretch{1}}
    {\Large\bfseries SENIOR DESIGN PROJECT\par}
    \vspace{1.5cm}
    {\Large ECE 186B\par}
    \vspace{1.5cm}
    {\Large\bfseries Design of Automated Hydroponics with IoT Capabilities\par}
    \vspace{3cm}
    {\large California State University, Fresno\par}
    {\large Lyles College of Engineering\par}
    {\large Electrical and Computer Engineering Department\par}
    \vspace{2cm}
    {\large Dr. Kriehn\par}
    {\large Professor Moore\par}
    \vspace{2cm}
    {\large BY\par}
    \vspace{1cm}
    {\large Liam Goss\par}
    {\large Luigi Santiago-Villa\par}
    \vspace{\stretch{2}}
\end{titlepage}
% Overview
\section*{Overview}
\addcontentsline{toc}{section}{Overview}
The hydroponics system has been successfully developed with a focus on user-friendliness, comprehensive data analytics, and internet connectivity. Functioning seamlessly as a household appliance, the initiative effectively addresses the crucial need for access to nutritious food in areas identified as food deserts. The automated system comprises multiple sensors for monitoring temperature, humidity, light intensity, and water pH, alongside a UV lighting system, water reservoir, water pump, and a microcontroller. These integrated components provide end-users with real-time insights into plant health and nutrient metrics.

Additionally, a web application has been developed to facilitate user interaction, enabling the sharing of insights, posing queries, and exchanging settings presets. Serving as the central nexus for accessing sensor data and engaging with community-driven inputs, this platform significantly enhances user experience. Notably, the solution stands out from existing market offerings by its meticulous design tailored to address the unique challenges of food deserts and foster reliance on local communities through social networking.

The project, spanning two college semesters and totaling nine months, adhered to a budget allocation of approximately \$500. Major milestones were achieved through meticulous planning and execution, encompassing design and implementation phases, further subdivided into research, component sourcing, component testing, and final integration. The resulting fully functional prototype, constructed from Oriented Strand Board (OSB), incorporates the aforementioned sensors, a microcontroller, lighting apparatus, and water management controls. The completion of a fully operational prototype was accomplished early in the project timeline, allowing for the cultivation of basic crops to vividly demonstrate the efficacy and functionality of the design.

\comment{


% Dedication (Optional)
\section*{Dedication}
\addcontentsline{toc}{section}{Dedication}
\lipsum[1] % Replace with your dedication text


}

\comment {


% Acknowledgment (Optional)
\section*{Acknowledgment}
\addcontentsline{toc}{section}{Acknowledgment}
\lipsum[1] % Replace with your acknowledgment text

}

% Table of Contents
\tableofcontents
\pagebreak

% List of Tables
\listoftables
\pagebreak

% List of Figures
\listoffigures
\pagebreak

% Abstract
\section*{Abstract}
\addcontentsline{toc}{section}{Abstract}
\noindent This project addresses food deserts by developing an accessible, internet-connected hydroponics system, enabling urban households to grow nutritious food. Leveraging temperature, humidity, light intensity, and water pH sensors, alongside UV lighting and a microcontroller, the system facilitates real-time plant health monitoring. A unique feature is its community-driven web application, allowing users to exchange growing tips and presets, significantly lowering the entry barrier for novices. Designed with a focus on food desert challenges and community engagement, the project aims to make home-grown crops viable for everyone. Over nine months and with a \$500 budget, this initiative will produce a functional prototype that demonstrates efficient crop cultivation, highlighting a technology-driven, socially inclusive approach to mitigate food insecurity in urban areas.


% Narrative Sections (Example structure)
\section{Narrative - what to title this?}
\noindent  Words go here

\subsection{Circuit Design}
\noindent

\begin{figure}[H]
    \centering
    \includegraphics[width=0.8\textwidth]{images/Schematic.jpg}
    \caption{Circuit Schematic Overview}
    \label{fig:Schematic_Full}
\end{figure}

\noindent The hydroponics system's circuit design was specially crafted to ensure the seamless integration of its diverse components. These include multiple sensors, a microcontroller, a water pump, and a UV lighting system, each fulfilling a distinct role within the automated hydroponic framework. The design facilitates real-time monitoring of environmental conditions, nutrient levels, and plant health metrics, empowering users to optimize growth conditions and ensure robust crop development.
\newline
\newline
\noindent Comprehensive sensor integration encompasses temperature, humidity, light intensity, and water pH, enabling thorough data collection crucial for plant growth assessment. The microcontroller acts as the system's core, gathering sensor data, managing component operations, and supporting user interaction through a web interface. Essential components like the water pump and UV lighting system play vital roles in sustaining optimal growth conditions by ensuring adequate hydration and nutrient distribution.
\newline
\newline
\noindent The circuit design was meticulously planned to accommodate diverse power demands, establish effective communication protocols between sensors and the microcontroller, and enable seamless data exchange for real-time monitoring and control. This systematic approach resulted in a fully operational automated growing environment. The resulting circuit configuration forms a sturdy foundation for monitoring and managing key parameters essential for successful plant cultivation, allowing users to foster healthy crops with minimal manual intervention.
\newline


\noindent 1. \textbf{Microcontroller}
\begin{itemize}
    \item Raspberry Pi Pico W
\end{itemize}

\noindent 2. \textbf{Water Pump}
\begin{itemize}
    \item CrocSee 12VDC Diaphragm Pump
\end{itemize}

\noindent 3. \textbf{Power Supply Unit (PSU)} 
\begin{itemize}
    \item BMOUO 12V AC-to-DC Power Supply 
    \item FILSHU 10A 250V Power Socket Inlet Switch
\end{itemize}

\noindent 4. \textbf{Buck Converter}
\begin{itemize}
    \item DORHEA C120503 12V to 5V DC Converter
\end{itemize}

\noindent 5. \textbf{Sensors}
   \begin{itemize}
   \item Four DHT22 sensors for temperature and humidity monitoring
   \item LTR390 UV Light Sensor for measuring UV light intensity
   \item HC-SR04 ultrasonic sensor
   \item PH-4502C analog pH sensor
   \end{itemize}

\noindent 6. \textbf{Additional Circuit Parts}
   \begin{itemize}
   \item Flyback diode (1N4001RLGOSCT-ND)
   \item 220µF Capacitor (16ZLH220MEFCT16.3X11)
   \item Two N-Channel MOSFET (IRLZ34NPBF-ND)
   \item 8:1 MUX (SN74LS151N)
   \item 2.42" SSD1309 OLED display
   \item 60W full spectrum grow lights
   \end{itemize}


\subsection{Server Configuration}
\noindent The central server is configured to hose the web application and manage the data 
from the hydroponics systems. The server runs on a Debian operating system. Debian is a popular and widely used Linux distribution known for its reliability and extensive package management system. Debian offers a wide range of software packages through its repositories. It is favored for server environments due to its reliability and security features, making it a suitable choice for hosting various web applications, services, and data management systems.
\newline
\newline
\noindent Flask is a simple and flexible framework for building web applications using Python. It helps developers create websites by providing tools for handling web page requests, organizing URLs, and displaying content. In Flask, routes serve as mappings between specific URLs and Python functions, enabling the application to respond to different requests. You can define custom functions for each URL endpoint, allowing dynamic content generation based on user interactions. Using the \verb|@app.route()| decorator, you specify the URL pattern associated with a particular function. Dynamic URLs can include variable parts, passed as arguments to the associated function, facilitating personalized responses. Within these functions, you can implement logic to generate dynamic content, such as fetching data from databases or processing user inputs. This approach enables the creation of dynamic web applications where each URL endpoint serves specific functionalities, thus offering a tailored user experience.
\newline
\newline
\noindent To enhance network reliability and accessibility for remote services, the deployment and configuration of the No-IP Dynamic Update Client (DUC) were undertaken on the Debian-based system. No-IP is a dynamic DNS service that allows internet users to provide a fixed domain name to their dynamically changing IP address, thereby ensuring that the connection to the server remains uninterrupted despite changes in the network configuration. This is particularly useful for hosting servers, remote access, and other network services that require constant accessibility over the internet. To integrate No-IP's DUC into the system's services for automatic startup and recovery, a systemd service unit file was meticulously crafted and deployed. This configuration specifies the execution of the No-IP DUC software as a background service, thereby enabling it to update DNS records automatically whenever the system's IP address changes. Through careful examination and troubleshooting of systemd service file settings, potential issues were identified and resolved, ensuring seamless operation. The implementation enhances the system's network reliability and facilitates uninterrupted remote access.
\newline
\newline
\noindent The remote server has two ports exposed to the internet, one for the web application and one for SSH. To reduce the risk of unauthorized access, the SSH port was changed from the default port 22 to a non-standard port. This simple security measure significantly reduces the number of unauthorized login attempts and enhances the system's overall security. The web application port was also changed to a nonstandard port to further reduce the risk of unauthorized access. In addition to altering the default ports for SSH and the web application, two additional security measures were implemented on the remote server: fail2ban and ufw.
\newline
\newline
\noindent Fail2ban is an intrusion prevention software framework that operates by monitoring log files for patterns indicating unsuccessful login attempts or other malicious activity. Upon detection of such patterns, fail2ban dynamically updates firewall rules to block the IP addresses associated with the detected activity. By effectively blocking malicious actors attempting unauthorized access, fail2ban enhances the server's security posture and mitigates the risk of successful brute-force attacks.
\newline
\newline
\noindent UFW, or Uncomplicated Firewall, is a front-end for managing firewall rules in Linux-based systems. It provides a user-friendly interface for configuring firewall settings and managing network traffic. UFW simplifies the process of creating and maintaining firewall rules, enabling administrators to define access policies based on specific criteria such as IP addresses, ports, and protocols. By leveraging UFW to enforce firewall rules, administrators can restrict access to services running on the server, thereby reducing the attack surface and enhancing overall security.
\newline
\newline
\noindent By combining the use of non-standard ports with fail2ban and UFW, the security posture of the remote server is significantly bolstered against unauthorized access attempts and potential malicious activity. These measures collectively contribute to mitigating the risks associated with operating services accessible over the internet, safeguarding the confidentiality, integrity, and availability of the hosted resources.
\newline
\newline
\subsubsection{Sub-Subsection Example}
\noindent Words go here

\subsection{Component Testing}
\subsubsection{pH Sensor - PH4502C}
\begin{table}[H]
    \centering
    \begin{tabular}{|c|c|}
        \hline
        \textbf{pH} & \textbf{Voltage [V]} \\
        \hline
        4 & 3.071\\
        \hline
        7 & 2.535\\
        \hline
        10 & 2.066\\
        \hline
    \end{tabular}
    \caption{PH4502C pH to Voltage Conversion}
    \label{tab:PH4502C}
\end{table}
\begin{figure}[H]
    \centering
    \includegraphics[width=0.8\textwidth]{images/ph_from_amazon.jpg}
    \caption{PH4502C pH Sensor \cite{ref_ph_amazon}}
    \label{fig:PH4502C}
\end{figure}
\noindent The PH4502C is designed to output the following voltages shown in Table \ref{tab:PH4502C}.
This data was provided by the manufacturer and was used to calibrate the pH sensor. An extensive calibration process was conducted for the PH4502C sensor module to ensure accurate pH readings. The sensor, equipped with two potentiometers for offset and slope adjustments, was designed to output a voltage correlating to the detected pH level, with a specified range of 0-5V.
\newline
\newline
\noindent Initially, difficulties were encountered in the calibration process due to output voltage readings exceeding the safe input range of the Raspberry Pi Pico’s analog-to-digital converter (ADC). A voltage divider was implemented to attenuate the sensor's voltage output. Subsequent readings from the sensor module were consistently registering maximum ADC values, indicative of an over-voltage condition. It was concluded that the direct output from the sensor was unsuitable for direct interfacing with the Pico’s ADC.
Following the integration of a voltage divider into the circuit, the output was reduced to acceptable levels within the Pico's ADC range of 0-3.3V. However, the initial calibration attempt, while using the voltage divider, produced a reading that did not align with the standard pH buffer solution. Consequently, the calibration process was refined by removing the voltage divider and reading the output directly from the sensor module, which resulted in a more stable and accurate voltage reading.
\newline
\newline
\noindent The offset potentiometer was then carefully adjusted while the sensor was immersed in a pH 7 buffer solution. The voltage was monitored using a multimeter, and the potentiometer was tuned until the output voltage closely matched the expected 2.51V as indicated on the product page for a pH 7 solution. This calibration point was further verified through a MicroPython script, which confirmed a pH reading of 7, correlating to the observed voltage of 2.5V.
\newline
\newline
\noindent An observation was made during the calibration that a red LED indicator on the sensor module was activated when the voltage dropped below 4V. It was inferred that this LED served as an indicator for out-of-range voltage conditions. The sensor was adjusted to maintain the output voltage above this threshold to prevent the activation of the LED indicator.
Once the sensor output was adjusted to the correct voltage for a pH 7 solution, the calibration was deemed successful. The calibration state was documented as stable, and the expectation was set that, barring any significant handling or environmental changes, recalibration would not be necessary before each use. 
\newline
\newline



% Figures and Tables example
\begin{figure}[H]
    \centering
    \includegraphics[width=0.8\textwidth]{example-image}
    \caption{Example Figure}
    \label{fig:example}
\end{figure}

\begin{table}[H]
    \centering
    \begin{tabular}{|c|c|c|}
        \hline
        Column 1 & Column 2 & Column 3 \\
        \hline
        Item 1 & Item 2 & Item 3 \\
        \hline
    \end{tabular}
    \caption{Example Table}
    \label{tab:example}
\end{table}

% Conclusion
\section{Conclusion}
\noindent Words go here

% Appendices
\appendix
\section{Appendix Title}
\noindent Words go here

% References
\begin{thebibliography}{99}
    \bibitem{ref_ph_amazon} GAOHOU. (n.d.). PH0-14 Value Detect Sensor Module + PH Electrode Probe BNC For Arduino [Online]. Available: \url{https://www.amazon.com/dp/B0799BXMVJ?psc=1&ref=ppx_yo2ov_dt_b_product_details}
    % Repeat for each reference
\end{thebibliography}

\end{document}
